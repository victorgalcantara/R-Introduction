\documentclass{article}
\title{Programação e análise de dados com R}
\author{Victor Gabriel Alcantara \\ victorgalcantara@usp.br \\ PPGS/USP}
 \date{}
 
\usepackage{titlesec}
\titleformat*{\subsection}{\normalfont}
\usepackage[pdftex]{hyperref}

\begin{document}

\maketitle
Este minicurso visa oferecer noções gerais e operações básicas para importação, análise e exportação de dados com R (através do ambiente de desenvolvimento RStudio). É importante ter instalado o \href{https://brieger.esalq.usp.br/CRAN/}{R} e o \href{https://www.rstudio.com/products/rstudio/download}{RStudio} no computador. Os materiais para o curso estarão disponíveis em minha página do \href{https://github.com/victorgalcantara?tab=repositories}{GitHub}.

\section{Noções gerais: é de comer?}
\subsection{História R e RStudio}
\subsection{Input, processamento e output}
\subsection{Fluxo de trabalho no RStudio: janelas, linguagens suportadas e convenções para o desenvolvimento de códigos}
\subsection{Práticas comunitárias: help, Stack Overflow e GitHub}

\section{Mão na massa}
\subsection{Operações básicas, comandos e lógica}
\subsection{Classes de objetos: vetores, arrays, matrizes, listas e data.frames/tibbles}
\subsection{Subset: navegação em objetos}
\subsection{Funções e pacotes: tidyverse e ggplot}

\section{Dados estruturados}
\subsection{Importação de dados estruturados}
\subsection{Manuseio dos dados: funções fundamentais!}
\subsection{Estatísticas descritivas}
\subsection{Exportação de dados e resultados de análises}


\end{document}