\documentclass[a4paper,12pt]{article}

\title{\vspace{-3.5cm}Programação e análise de dados com R \\ \large Da introdução à autonomia intermediária}
\author{Victor Gabriel Alcantara \\ victorgalcantara@usp.br \\ PPGS/USP}
 \date{}

\usepackage{titlesec}
\titleformat*{\subsection}{\normalfont}
\usepackage[pdftex]{hyperref}
\begin{document}

\maketitle

Este curso aborda o funcionamento da linguagem de programação R e do software de desenvolvimento RStudio/Posit, visando oferecer noções gerais e operações básicas para o domínio e a autonomia na análise de dados com programação. O curso envolve operações de importação, análise, visualização, exportação e comunicação de dados, e está estruturado seguindo os eixos do livro de \href{https://r4ds.had.co.nz/}{Wickham e Grolemund (2018)}. Por seguir o livro de referência, os tópicos mesclam com o inglês para estimular a adaptação ao software, à comunidade de programadores internacionais e aos manuais, que têm a língua inglesa como padrão.

\vspace{1cm}

\textbf{Objetivo geral}: oferecer condições para a autonomia na programação e no uso do software, por isso os fundamentos são abordados com cuidado. É esperado que se saiba, ao final do curso, trabalhar com essa linguagem de programação para operar com o instrumental básico da Ciência de Dados, que no fundo são aplicações da estatística e econometria.

\vspace{0.5cm}

\textbf{Dinâmica}: o curso é online com aulas síncronas. As aulas serão acompanhadas de materiais que estarão disponibilizados em minha página do GitHub. Serão \underline{recomendados} exercícios para praticar o uso da linguagem, verificar dúvidas e exercitar a autonomia.

\vspace{0.5cm}

\textbf{Pré-requisitos teóricos}: interesse em aprender e trabalhar com análise de dados com programação. Não é exigido nenhum conhecimento em R, matemática, estatística muito menos inglês.

\vspace{0.5cm}

\textbf{Pré-requisitos técnicos}: É necessário um computador com configurações padrão (4GB RAM e estrutura 64bits) e uma conexão estável com a internet. É importante ter instalado o \href{https://brieger.esalq.usp.br/CRAN/}{R} e o \href{https://www.rstudio.com/products/rstudio/download}{RStudio} no computador. É recomendado a integração na plataforma \href{https://github.com/}{GitHub}, atualmente a mais utilizada por programadores (não se preocupe, será apresentada na aula inicial). Os materiais para o curso estarão disponíveis em minha página do \href{https://github.com/victorgalcantara?tab=repositories}{GitHub}.

\vspace{0.5cm}

\textbf{Inscrições}: até o dia 30 de abril pelo formulário 

%\vspace{1cm}

%\textbf{Sobre mim}: Cientista Social (IFCS/UFRJ) e mestre em Sociologia (PPGSA/UFRJ), atualmente cursando o doutorado em Sociologia (PPGS/USP) e o MBA em Ciência de Dados (ESALQ/USP). Comecei a me dedicar ao R em 2020, quando ingressei no mestrado. Cursei disciplinas de metodologia quantitativa e programação voltada para as Ciências Sociais no IESP/UERJ, trabalhei com o professor José G Dias (ISCTE/Lisboa) e continuo meus estudos no MBA e em disciplinas da pós na USP.

\section{Explore: basic knowledge and workflow}
\section*{Noções gerais: é de comer?}
\subsection*{História R e RStudio}
\subsection*{Input, processamento e output}
\subsection*{Fluxo de trabalho no RStudio: janelas, linguagens suportadas e convenções para o desenvolvimento de códigos}
\subsection*{Práticas comunitárias: help, Stack Overflow e GitHub}

\section{Explore: logic operations and basic R}
\section*{Mão na massa}
\subsection*{Operações básicas, comandos e lógica}
\subsection*{Classes de objetos: vetores, arrays, matrizes, listas e data.frames/tibbles}
\subsection*{Subset: navegação em objetos}
\subsection*{Funções e pacotes: tidyverse e ggplot}

\section{Import and tidy data}
\section*{Dados estruturados}
\subsection*{Import: importação de dados}
\subsection*{Tidy and Transform: manuseio dos dados (filter, select, rename, mutate, group by etc.)}

\pagebreak

\section{Analyse}
\section*{Estatísticas descritivas}

\subsection*{Variáveis categóricas nominais e ordinais}
\subsubsection*{Retomando factor}
\subsubsection*{Tabela de frequência}

\subsection*{Variáveis categóricas tipo likert}
\subsubsection*{Média}

\subsection*{Variáveis métricas discretas e contínuas}
\subsubsection*{Média, Mediana e Moda}

\section{Visualize}
\section*{Gráficos}

\subsection*{Exportação de dados e resultados de análises}
\subsection*{Variáveis categóricas nominais e ordinais}
\subsubsection*{Barras e Pizza}

\subsection*{Variáveis métricas discretas e contínuas}
\subsubsection*{Barras e histogramas}
\subsubsection*{Histogramas e densidade}

\pagebreak

\section*{Análise Bivariada}
\subsection*{Definição conceitual: o que significa analisar duas variáveis?}
\subsection*{Categórica e Categórica}
\subsubsection*{Tabela de contingência ou cruzada}
\subsubsection*{Estatística Qui-Quadrado}

\subsection*{Categórica e Métrica}
\subsubsection*{Barras, boxplot e densidades}
\subsubsection*{Diferença entre médias}

\subsection*{Métrica e Métrica}

\subsubsection*{Dispersão}
\subsubsection*{Correlação de Pearson}

\subsection*{Métodos para exportação dos resultados}

\section{Model}
\section*{Análise Multivariada}
\subsection*{Regressão linear simples pelo Método dos Mínimos Quadrados (MQO)}
\subsection*{Regressão linear múltipla pelo Método dos Mínimos Quadrados (MQO)}

\section{Communicate}
\section*{Rmarkdown}
\subsection*{Chunks}

\end{document}