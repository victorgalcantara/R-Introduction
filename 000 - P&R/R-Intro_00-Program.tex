\documentclass[a4paper,12pt]{article}

\usepackage{titlesec}
\titleformat*{\subsection}{\normalfont}
\usepackage[pdftex]{hyperref}
\usepackage[brazil]{babel}
\usepackage[style=nature]{biblatex} %Imports biblatex package
\usepackage{csquotes} 

\addbibresource{biblio.bib} %Import the bibliography file

% informações do PDF
\makeatletter
\hypersetup{
     	%pagebackref=true,
		pdftitle={\@title}, 
		pdfauthor={\@author},
    	pdfsubject={Programa Programação e análise de dados com R},
	    pdfcreator={LaTeX with abnTeX2},
		pdfkeywords={programação}{Ciência de Dados}{econometria}{estatística}{R}{RStudio}, 
		colorlinks=true,       		% false: boxed links; true: colored links
    	linkcolor=blue,          	% color of internal links
    	citecolor=blue,        		% color of links to bibliography
    	filecolor=magenta,      		% color of file links
		urlcolor=blue,
		bookmarksdepth=4
}
\makeatother
% --- 

\title{\vspace{-3.5cm}Programação e análise de dados com R \\ \large Da introdução à autonomia intermediária}
\author{Victor Gabriel Alcantara \\ victorgalcantara@usp.br \\ PPGS/USP}
\date{}

\begin{document}

\maketitle

O curso aborda o funcionamento da linguagem de programação R e do software de desenvolvimento RStudio/Posit, visando oferecer noções gerais e operações básicas para o domínio e a autonomia na análise de dados com programação. O curso envolve operações de importação, análise, visualização, exportação e comunicação de dados, e está estruturado seguindo os eixos do livro de \href{https://r4ds.had.co.nz/}{Wickham e Grolemund (2018)}\cite{wickham2017}. Os tópicos mesclam com o inglês para estimular a adaptação ao software, à comunidade de programadores internacionais e aos manuais, que têm a língua inglesa como padrão.

\vspace{1cm}

\textbf{Objetivo geral}: oferecer condições para a autonomia na programação com R e no uso do software RStudio. É esperado que se saiba, ao final do curso, trabalhar com R para operar com o instrumental básico da Ciência de Dados, que são aplicações da estatística e econometria.

\vspace{0.5cm}

\textbf{Dinâmica}: online com aulas síncronas e materiais disponibilizados em minha página do GitHub. Serão \underline{recomendados} exercícios para praticar o uso da linguagem, verificar dúvidas e exercitar a autonomia.

\vspace{0.5cm}

\textbf{Pré-requisitos teóricos}: interesse em aprender e trabalhar com análise de dados com programação. Não é exigido nenhum conhecimento em R, matemática, estatística muito menos inglês.

\vspace{0.5cm}

\textbf{Pré-requisitos técnicos}: É necessário um computador com configurações padrão (4GB RAM e estrutura 64bits) e uma conexão estável com a internet. É importante ter instalado o \href{https://brieger.esalq.usp.br/CRAN/}{R} e o \href{https://www.rstudio.com/products/rstudio/download}{RStudio} no computador. É recomendado a integração na plataforma \href{https://github.com/}{GitHub}, atualmente a mais utilizada por programadores (não se preocupe, será apresentada na aula inicial). Os materiais para o curso estarão disponíveis em minha página do \href{https://github.com/victorgalcantara?tab=repositories}{GitHub}.

\vspace{0.5cm}

\textbf{Inscrições}: até o dia 30 de abril pelo formulário 

%\vspace{1cm}

%\textbf{Sobre mim}: Cientista Social (IFCS/UFRJ) e mestre em Sociologia (PPGSA/UFRJ), atualmente cursando o doutorado em Sociologia (PPGS/USP) e o MBA em Ciência de Dados (ESALQ/USP). Comecei a me dedicar ao R em 2020, quando ingressei no mestrado. Cursei disciplinas de metodologia quantitativa e programação voltada para as Ciências Sociais no IESP/UERJ, trabalhei com o professor José G Dias (ISCTE/Lisboa) e continuo meus estudos no MBA e em disciplinas da pós na USP.

\section{Explore: basic knowledge and workflow}
\section*{Noções gerais: é de comer?}

\begin{itemize}
    \item História R e RStudio \cite{ihaka1996}\cite{ihaka1998}

    \item Input, processamento e output

    \item Fluxo de trabalho no RStudio: janelas, linguagens suportadas e convenções para o desenvolvimento de códigos

    \item Práticas comunitárias: help, Stack Overflow e GitHub
\end{itemize}


\section{Explore: logic operations and basic R}
\section*{Mão na massa}

\begin{itemize}
    \item Operações básicas, comandos e lógica

    \item Classes de objetos: vetores, arrays, matrizes, listas e data.frames

    \item Subset: navegação em objetos

    \item Funções e pacotes: tidyverse e ggplot
\end{itemize}

\section{Import and tidy data}
\section*{Dados estruturados}

\begin{itemize}
    \item Import: importação de dados

    \item Tidy and Transform: manuseio dos dados (filter, select, rename, mutate, group by etc.)

\end{itemize}

\pagebreak

\section{Analyse} \nocite{angrist_mastering_2015}\nocite{imai_quantitative_2017}
\section*{Estatísticas descritivas}

\begin{itemize}
    \item Variáveis categóricas nominais e ordinais

    \item[-] Tabela de frequência

    \item Variáveis categóricas tipo likert

    \item[-] Média

     \item Variáveis métricas discretas e contínuas

     \item[-] Média, Mediana e Moda

\end{itemize}

\section{Visualize}
\section*{Gráficos}
\subsection*{GGplot: A Grammar of Graphics \cite{wickham2016}}
\begin{itemize}
    \item Variáveis categóricas nominais e ordinais

    \item[-] Barras e Pizza

     \item Variáveis métricas discretas e contínuas

     \item[-] Barras, boxplot, histogramas e densidade
\end{itemize}

\pagebreak

\section*{Análise Bivariada}

\begin{itemize}
    \item Definição conceitual: o que significa analisar duas variáveis?

    \item Categórica e Categórica

    \item[-] Tabela de contingência ou cruzada

    \item[-] Estatística Qui-Quadrado e V de Cramer

     \item Categórica e Métrica

     \item[-] Boxplot, histogramas, densidade e Diferença entre médias

     \item Métrica e Métrica
     \item[-] Dispersão
     \item[-] Correlação de Pearson
     
\end{itemize}

\section{Model}
\section*{Análise Multivariada}


\begin{itemize}
    \item Regressão linear simples pelo Método dos Mínimos Quadrados (MQO)
     
     \item Regressão linear múltipla pelo Método dos Mínimos Quadrados (MQO)
     
\end{itemize}

\section{Communicate}
\section*{Rmarkdown}

\printbibliography %Prints bibliography

\end{document}
